%%%% 一个建议的幻灯片环境
\startenvironment % 与第一行代码间必须有空行 BUG

% %%%%%%%%%%%%%%%%%%%%%%%%%%%%%%%%%%%%%%%%%%%%%%
% 模式设置
% %%%%%%%%%%%%%%%%%%%%%%%%%%%%%%%%%%%%%%%%%%%%%%

% 显示所有步骤,高亮当前步骤(默认为显示以前步骤,隐藏后续步骤)
\definemode[highlightcurrentstep]
% \enablemode[highlightcurrentstep]
\disablemode[highlightcurrentstep]

% 在分步中(条件判断用)
\definemode[insteps]
% \enablemode[insteps]
\disablemode[insteps]

% %%%%%%%%%%%%%%%%%%%%%%%%%%%%%%%%%%%%%%%%%%%%%%
% 配置
% %%%%%%%%%%%%%%%%%%%%%%%%%%%%%%%%%%%%%%%%%%%%%%
% 中文配置
\mainlanguage[cn]
\language[cn]
\setscript[hanzi] % 汉字处理脚本(断行等)
\setupalign[hanging,hz] %行末标点悬挂

% 页面和用纸的尺寸
\setuppapersize[SW][SW]
\setuppapersize[S6][S6]
% [S3] [width=300pt,height=225pt]
% [S4] [width=400pt,height=300pt]
% [S5] [width=500pt,height=375pt]
% [S6] [width=600pt,height=450pt]
% [S8] [width=800pt,height=600pt]
% [SW] [width=800pt,height=450pt]
% [SM] [width=720pt,height=450pt]

% 定义页面为装饰页
\definemakeup[slidepage]
\definelayout[slidepage][width=fit,
    rightmargin=1.5cm,
    leftmargin=1.5cm,
    leftmargindistance=0pt,
    rightmargindistance=0pt,
    topspace=0cm,
    height=fit,
        header=0pt,
        footer=1em,
    bottomspace=0cm,
    backspace=1.5cm,
    bottom=12pt,
    location=singlesided]
\setupmakeup[slidepage][doublesided=empty, pagestate=start, align=middle, page={disable,no}]

% 颜色设置
% \setupcolors[state=start]
\usecolors[crayola]%使用颜色集合,色标见:https://wiki.contextgarden.net/Color
\definecolor[transparentColor][k=0,t=0,a=1] % 透明色(就地隐藏文本用)
% \definecolor[highlightColor][BlueI] % 高亮色(显示当前步骤)
\definecolor[highlightColor][white] % 高亮色(显示当前步骤)

% 背景颜色
\setupbackgrounds[page][background=color,backgroundcolor=Inchworm]
    % {\setupbackgrounds[page][background=color,backgroundcolor=white]}
% \startcolor[OuterSpace]
% \startcolor[white]

% 页码
% \setuppagenumber[state=stop]
\setuppagenumbering[location={footer,right}]
% \setuptolerance[verytolerant,stretch] 
\setupinteractionscreen[option=max]


% 标题设置
\setupheads[chapter][
    before=,
    after=,
    number=no,
    page=no,% 单元设置,新页起 , page=yes新页/left偶数页/right奇数页(默认)
    style={\setupalign[middle]\tfb},
    ]
\setuplabeltext[cn][chapter={{},{、}}] % 前后标签

\setupindenting[no]

% %%%%%%%%%%%%%%%%%%%%%%%%%%%%%%%%%%%%%%%%%%%%%%
% 交互配置:目录、PDF书签、交互、引用
% %%%%%%%%%%%%%%%%%%%%%%%%%%%%%%%%%%%%%%%%%%%%%%
\setupheadtext[content=目录]
\setupcombinedlist[content][list={DANYUAN,KEWEN,lanmuN},alternative=c]%
\setuplist[DANYUAN][style=bold]
\setuplist[KEWEN][margin=2em]
% pdf交互/链接
\setupinteraction[state=start,color=,]
% pdf书签
% \setupinteractionscreen[option={bookmark}]
\setupinteractionscreen[
    option={doublesided,bookmark},
    width=max,height=max,% necessary for Trim/BleedBox
]
\placebookmarks[DANYUAN,KEWEN,lanmuN,chapter,section,subsection] % 可加入自定义\definelist[mylist]
% 设置引用
% \setupreferencing[left=,right=] % 覆盖左右两侧的引号
% \define[1]\see{\at{}{页}[#1]\about[#1]}% 另见

% %%%%%%%%%%%%%%%%%%%%%%%%%%%%%%%%%%%%%%%%%%%%%%
% 定义排版命令
% %%%%%%%%%%%%%%%%%%%%%%%%%%%%%%%%%%%%%%%%%%%%%%
% 标题
\def\sTitle#1{\title{\tfb #1}}
\def\sChapter#1{\doifmodeelse{insteps}%
    {\sTitle{#1}}% 无序号标题(不上目录)
    {\chapter{\tfb #1}}}% 有序号标题(上目录)

% 设置列表组itemize、所有级别、阿拉伯数字、堆叠、其后不留空
\setupitemgroup[itemize][each][packed][before={}, after={},]
% 可控制宽度的列表组
\def\sItems[#1]#2{\centeraligned{\framedtext[before=,offset=0pt,width=#1,frame=off]{\startitemize[packed] #2 \stopitemize}}}
\def\sPar[#1]#2{\centeraligned{\framedtext[before=,after=,offset=0pt,width=#1,frame=off,indenting={}]{#2}}}

% 分步显示(保持所有元素占位,仅通过显隐或高亮来切换)
\definecounter[recursenum][way=bytext]
% \setnumber[recursenum][1]
\definecounter[currentstepnum][way=bytext]
% \setnumber[currentstepnum][1]
\def\step[#1]#2{%
    \doifmodeelse{highlightcurrentstep}%
    {\ifnum \rawcountervalue[currentstepnum]=1 \color[transparentColor]{#2} \else%
        \ifnum #1=\rawcountervalue[currentstepnum]\color[highlightColor]{#2} \else #2 \fi%
    \fi}%
    {\ifnum #1>\rawcountervalue[currentstepnum]\color[transparentColor]{#2} \else #2 \fi}%
    % 
    \ifnum #1>\rawcountervalue[recursenum] \setnumber[recursenum][#1] \fi%收集最大步骤
    }

% 仅在预定预定步骤排印
\def\onstep[#1]#2{%
    \ifnum #1=\rawcountervalue[currentstepnum]#2\fi%
    \ifnum #1>\rawcountervalue[recursenum] \setnumber[recursenum][#1] \fi%收集最大步骤
    }

% 在预定步骤后排印
\def\fromstep[#1]#2{%
    \ifnum \numexpr\rawcountervalue[currentstepnum]+1\relax>#1 #2\fi%
    \ifnum #1>\rawcountervalue[recursenum] \setnumber[recursenum][#1] \fi%收集最大步骤
    }

% 幻灯片环境
\definestartstop[Slide]
[commands={\switchtobodyfont[24pt]},
before={\startmakeup[slidepage]},
after={\stopmakeup}]
\def\slide#1{%
    \setnumber[recursenum][1]%
    \setnumber[currentstepnum][1]%
    \startSlide #1 \stopSlide%
    \decrementcounter[recursenum]%
    \incrementcounter[currentstepnum]%
    % 
    \enablemode[insteps]%
    \dorecurse{\rawcountervalue[recursenum]}{%
        \startSlide #1 \stopSlide%
    \incrementcounter[currentstepnum]}%
    \setnumber[recursenum][1]%
    \setnumber[currentstepnum][1]%
    \disablemode[insteps]%
}

% 无宽度(不影响对齐)的单行小字注释,适用于行末
\def\noteonstep[#1]#2{\onstep[#1]{\hbox to 0pt{\itxx #2}}}
\def\notefromstep[#1]#2{\fromstep[#1]{\hbox to 0pt{\itxx #2}}}

\stopenvironment